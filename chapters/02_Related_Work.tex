% %!TEX root = ../main.tex
\chapter{Related Work}
\label{chp:relatedWork}

With the rise of the Internet, there has been a notable surge in digital text creation across various platforms such as social media, emails, blogs, news articles, publications, and online forums. This vast corpus of unstructured or semi-structured text harbors a wealth of information. Information Extraction (IE) is a pivotal tool in discerning and organizing meaningful insights from these textual sources, transforming them into structured data.

One way to represent information in the text is in the form of entities and relations representing links between entities. Therefore, Named Entity Recognition (NER) and Relation Extraction (RE) emerge as particularly valuable techniques and key components of IE. They enable the extraction of pertinent entities and relationships within the text, facilitating the conversion of raw data into structured repositories of valuable information.

The NER identifies entities from the text, and the RE task can identify relationships between those entities. Furthermore, end-to-end relation extraction aims to identify named entities and extract relations between them in one go. Effectively modeling these two subtasks jointly\cite{Zhong2020AFE}, either by casting them in one structured prediction framework, or performing multi-task learning through shared representations.

Many NLP applications can benefit from relational information derived from natural language\cite{Goyal2018RNE}, including Structured Search, Knowledge Base (KB) population, Information Retrieval, Question-Answering, Language Understanding, Ontology Learning, etc. Therefore these tasks have been studied extensively and many models have been proposed to tackle them.
\section{Section 1}

\section{Section 2}


